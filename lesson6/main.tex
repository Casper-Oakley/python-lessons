% Copyright 2004 by Till Tantau <tantau@users.sourceforge.net>.
%
% In principle, this file can be redistributed and/or modified under
% the terms of the GNU Public License, version 2.
%
% However, this file is supposed to be a template to be modified
% for your own needs. For this reason, if you use this file as a
% template and not specifically distribute it as part of a another
% package/program, I grant the extra permission to freely copy and
% modify this file as you see fit and even to delete this copyright
% notice. 

\documentclass{beamer}
% Replace the \documentclass declaration above
% with the following two lines to typeset your 
% lecture notes as a handout:
%\documentclass{article}
%\usepackage{beamerarticle}

\usepackage{graphicx}
\usepackage[utf8]{inputenc}
\usepackage{tabto}
 
\graphicspath{ {img/} }


% There are many different themes available for Beamer. A comprehensive
% list with examples is given here:
% http://deic.uab.es/~iblanes/beamer_gallery/index_by_theme.html
% You can uncomment the themes below if you would like to use a different
% one:
%\usetheme{AnnArbor}
%\usetheme{Antibes}
%\usetheme{Bergen}
%\usetheme{Berkeley}
%\usetheme{Berlin}
%\usetheme{Boadilla}
%\usetheme{boxes}
%\usetheme{CambridgeUS}
%\usetheme{Copenhagen}
%\usetheme{Darmstadt}
%\usetheme{default}
%\usetheme{Frankfurt}
%\usetheme{Goettingen}
%\usetheme{Hannover}
%\usetheme{Ilmenau}
%\usetheme{JuanLesPins}
%\usetheme{Luebeck}
%\usetheme{Madrid}
%\usetheme{Malmoe}
%\usetheme{Marburg}
%\usetheme{Montpellier}
%\usetheme{PaloAlto}
%\usetheme{Pittsburgh}
%\usetheme{Rochester}
%\usetheme{Singapore}
%\usetheme{Szeged}
\usetheme{Warsaw}

\title{Programming with Python}

% A subtitle is optional and this may be deleted
\subtitle{Lesson 6: PyGame and Flask!}

%\author{F.~Author\inst{1} \and S.~Another\inst{2}}
% - Give the names in the same order as the appear in the paper.
% - Use the \inst{?} command only if the authors have different
%   affiliation.

%\institute[Universities of Somewhere and Elsewhere] % (optional, but mostly needed)
%{
%  \inst{1}%
%  Department of Computer Science\\
%  University of Somewhere
%  \and
%  \inst{2}%
%  Department of Theoretical Philosophy\\
%  University of Elsewhere}
% - Use the \inst command only if there are several affiliations.
% - Keep it simple, no one is interested in your street address.

\date{December 6th, 2016}
% - Either use conference name or its abbreviation.
% - Not really informative to the audience, more for people (including
%   yourself) who are reading the slides online

\subject{Python Lessons}
% This is only inserted into the PDF information catalog. Can be left
% out. 

% If you have a file called "university-logo-filename.xxx", where xxx
% is a graphic format that can be processed by latex or pdflatex,
% resp., then you can add a logo as follows:

% \pgfdeclareimage[height=0.5cm]{university-logo}{university-logo-filename}
% \logo{\pgfuseimage{university-logo}}

% Delete this, if you do not want the table of contents to pop up at
% the beginning of each subsection:
%\AtBeginSubsection[]
%{
%  \begin{frame}<beamer>{Outline}
%    \tableofcontents[currentsection,currentsubsection]
%  \end{frame}
%}

% Let's get started
\begin{document}

\begin{frame}
  \titlepage
\end{frame}

%\begin{frame}{Outline}
%  \tableofcontents
%  % You might wish to add the option [pausesections]
%\end{frame}

% Section and subsections will appear in the presentation overview
% and table of contents.
\section{Introduction \& Recap}

\begin{frame}{Last week's goals}
\pause
  \begin{itemize}
  \item We learnt about tuples and their uses in returning multiple values\pause
  \item We learnt about dictionaries\pause
  \item We discussed objects\pause
  \item We finished writing our own text based game!
  \end{itemize}  
\end{frame}

\section{Flask}

\begin{frame}{Python in action: Flask}
Flask is a web framework, written in python allowing us to write web apps in python.\\ \pause

We can use it to write interactive websites or neat web services.
\end{frame}

\begin{frame}{Getting started with flask - Setting up our environment}

\pause
Folder structure:
\begin{itemize}
  \item[] my-awesome-flask-app/\pause \qquad root directory of code
  \item[] my-awesome-flask-app/views/\pause \qquad directory of html to be rendered
  \item[] my-awesome-flask-app/public/\pause \qquad directory of static files e.g. css \& js
  \item[] my-awesome-flask-app/app.py\pause \qquad where our code will start
\end{itemize}

\end{frame}

\begin{frame}{Getting started with flask - Setting up our environment}

To install flask - \textit{pip install flask}

\end{frame}

\begin{frame}{Hello World!}

\begin{figure}[h]
\includegraphics[width=0.9\textwidth]{flask}
\end{figure}

\end{frame}
\begin{frame}{Hello World!}

\begin{figure}[h]
\includegraphics[width=0.9\textwidth]{flask1}
\end{figure}

\end{frame}
\begin{frame}{Hello World!}

\begin{figure}[h]
\includegraphics[width=0.9\textwidth]{flask2}
\end{figure}

\end{frame}
\begin{frame}{Hello World!}

\begin{figure}[h]
\includegraphics[width=0.9\textwidth]{flask3}
\end{figure}

\end{frame}
\begin{frame}{Hello World!}

\begin{figure}[h]
\includegraphics[width=0.9\textwidth]{flask4}
\end{figure}

\end{frame}

\begin{frame}{A side point on HTTP}

\pause
GET vs POST (vs others)
\pause
Status code responses\pause
\begin{itemize}
  \item 200\pause
  \item 304\pause
  \item 400\pause
  \item 401\pause
  \item 403\pause
  \item 404\pause
  \item 500
  \end{itemize}

\end{frame}

\begin{frame}{Endpoints}

An endpoint is a \textit{destination} 


\end{frame}

\section{Summary}

\begin{frame}{That's all for tonight}
  To summarise:
  \pause
  \begin{itemize}
  \item We learnt about tuples and their uses in returning multiple values\pause
  \item We learnt about dictionaries\pause
  \item We discussed objects\pause
  \item We finished writing our own text based game!
  \end{itemize}
\end{frame}

\begin{frame}{For next week}
Source code plus lecture slides will be available online soon after the lesson.\\
If you are new to HackSocNotts, please join us on \textit{http://hacksocnotts.slack.com}.\\
If you have any questions, feel free to ask now or over slack.\\
\end{frame}

\end{document}


