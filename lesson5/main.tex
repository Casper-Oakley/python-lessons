% Copyright 2004 by Till Tantau <tantau@users.sourceforge.net>.
%
% In principle, this file can be redistributed and/or modified under
% the terms of the GNU Public License, version 2.
%
% However, this file is supposed to be a template to be modified
% for your own needs. For this reason, if you use this file as a
% template and not specifically distribute it as part of a another
% package/program, I grant the extra permission to freely copy and
% modify this file as you see fit and even to delete this copyright
% notice. 

\documentclass{beamer}
% Replace the \documentclass declaration above
% with the following two lines to typeset your 
% lecture notes as a handout:
%\documentclass{article}
%\usepackage{beamerarticle}

\usepackage{graphicx}
\usepackage[utf8]{inputenc}
\usepackage{tabto}
 
\graphicspath{ {img/} }


% There are many different themes available for Beamer. A comprehensive
% list with examples is given here:
% http://deic.uab.es/~iblanes/beamer_gallery/index_by_theme.html
% You can uncomment the themes below if you would like to use a different
% one:
%\usetheme{AnnArbor}
%\usetheme{Antibes}
%\usetheme{Bergen}
%\usetheme{Berkeley}
%\usetheme{Berlin}
%\usetheme{Boadilla}
%\usetheme{boxes}
%\usetheme{CambridgeUS}
%\usetheme{Copenhagen}
%\usetheme{Darmstadt}
%\usetheme{default}
%\usetheme{Frankfurt}
%\usetheme{Goettingen}
%\usetheme{Hannover}
%\usetheme{Ilmenau}
%\usetheme{JuanLesPins}
%\usetheme{Luebeck}
%\usetheme{Madrid}
%\usetheme{Malmoe}
%\usetheme{Marburg}
%\usetheme{Montpellier}
%\usetheme{PaloAlto}
%\usetheme{Pittsburgh}
%\usetheme{Rochester}
%\usetheme{Singapore}
%\usetheme{Szeged}
\usetheme{Warsaw}

\title{Programming with Python}

% A subtitle is optional and this may be deleted
\subtitle{Lesson 5: Dictionaries and Tuples!}

%\author{F.~Author\inst{1} \and S.~Another\inst{2}}
% - Give the names in the same order as the appear in the paper.
% - Use the \inst{?} command only if the authors have different
%   affiliation.

%\institute[Universities of Somewhere and Elsewhere] % (optional, but mostly needed)
%{
%  \inst{1}%
%  Department of Computer Science\\
%  University of Somewhere
%  \and
%  \inst{2}%
%  Department of Theoretical Philosophy\\
%  University of Elsewhere}
% - Use the \inst command only if there are several affiliations.
% - Keep it simple, no one is interested in your street address.

\date{November 29th, 2016}
% - Either use conference name or its abbreviation.
% - Not really informative to the audience, more for people (including
%   yourself) who are reading the slides online

\subject{Python Lessons}
% This is only inserted into the PDF information catalog. Can be left
% out. 

% If you have a file called "university-logo-filename.xxx", where xxx
% is a graphic format that can be processed by latex or pdflatex,
% resp., then you can add a logo as follows:

% \pgfdeclareimage[height=0.5cm]{university-logo}{university-logo-filename}
% \logo{\pgfuseimage{university-logo}}

% Delete this, if you do not want the table of contents to pop up at
% the beginning of each subsection:
%\AtBeginSubsection[]
%{
%  \begin{frame}<beamer>{Outline}
%    \tableofcontents[currentsection,currentsubsection]
%  \end{frame}
%}

% Let's get started
\begin{document}

\begin{frame}
  \titlepage
\end{frame}

%\begin{frame}{Outline}
%  \tableofcontents
%  % You might wish to add the option [pausesections]
%\end{frame}

% Section and subsections will appear in the presentation overview
% and table of contents.
\section{Introduction \& Recap}

\begin{frame}{Last week's goals}
\pause
  \begin{itemize}
  \item We learnt about lists\pause
  \item We learnt about how we can iterate on lists with a for loop\pause
  \item We learnt about some list operations\pause
  \item We begun writing our own text based game!
  \end{itemize}
  
\end{frame}

\section{Tuples}

\begin{frame}{A flashback to lists}

Lists - A collection of anything you want\\ \pause

Can be resized\\ \pause

Elements can be changed when we want

\end{frame}

\begin{frame}{Tuples}

Tuples - A collection of anything you want\\ \pause

Cannot be resized\\ \pause

Once made, elements cannot be changed - You will not be able to do for example x[0] = 4. \pause This means tuples are \textit{immutable}.

\end{frame}

\begin{frame}{Tuple syntax}

\begin{figure}[h]
\includegraphics[width=0.8\textwidth]{tuple}
\end{figure}
\pause
\begin{figure}[h]
\includegraphics[width=0.8\textwidth]{tuple2}
\end{figure}


\end{frame}


\begin{frame}{What's the point?}

As tuples are immutable, they are much faster.\\ \pause

We can also use tuples if we want to return multiple values from a function.

\end{frame}

\begin{frame}{Example involving functions}

\begin{figure}[h]
\includegraphics[width=0.99\textwidth]{tuplefunc}
\end{figure}
\pause
Note the neat syntax on the left of the equals sign. You can kind of 'map' variables to the locations in a tuple.
\end{frame}

\section{Dictionaries}

\begin{frame}{What is a dictionary}

Dictionaries are big 'lists' of key-value pairs.\\ \pause
Keys can be ints, floats, strings, or tuples. Values can be absolutely anything.\\ \pause
They have plenty of uses, including:

  \begin{itemize}
  \item Storing place name to coordinates for a GPS tracking app\pause
  \item Storing weapon names to weapon stats for a game\pause
  \item Translating something from english to german
  \end{itemize}

\end{frame}

\begin{frame}{An example of a dictionary}

\begin{figure}[h]
\includegraphics[width=0.85\textwidth]{dict}
\end{figure}

\end{frame}

\section{More Games}

\section{Summary}

\begin{frame}{That's all for tonight}
  To summarise:
  \pause
  \begin{itemize}
  \item We learnt about lists\pause
  \item We learnt about how we can iterate on lists with a for loop\pause
  \item We learnt about some list operations\pause
  \item We begun writing our own text based game!
  \end{itemize}
\end{frame}

\begin{frame}{For next week}
Source code plus lecture slides will be available online soon after the lesson.\\
If you are new to HackSocNotts, please join us on \textit{http://hacksocnotts.slack.com}.\\
If you have any questions, feel free to ask now or over slack.\\
\end{frame}

\end{document}


